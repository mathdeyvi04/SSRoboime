\chapter{Descrição}
\hypertarget{md_src_2cpp_2README}{}\label{md_src_2cpp_2README}\index{Descrição@{Descrição}}
\label{md_src_2cpp_2README_autotoc_md0}%
\Hypertarget{md_src_2cpp_2README_autotoc_md0}%
 Nesta pasta, nos dedicaremos aos trabalhos que necessitam de uma maior perfomance.

Para padronizarmos o conteúdo dessas pastas, faremos o seguinte\+:

\tabulinesep=1mm
\begin{longtabu}spread 0pt [c]{*{2}{|X[-1]}|}
\hline
\cellcolor{\tableheadbgcolor}\textbf{ Arquivos Obrigatórios   }&\cellcolor{\tableheadbgcolor}\textbf{ Responsabilidades    }\\\cline{1-2}
\endfirsthead
\hline
\endfoot
\hline
\cellcolor{\tableheadbgcolor}\textbf{ Arquivos Obrigatórios   }&\cellcolor{\tableheadbgcolor}\textbf{ Responsabilidades    }\\\cline{1-2}
\endhead
Makefile   &Automatizará compilação individual    \\\cline{1-2}
debug.\+cc   &Proverá ferramentas de teste    \\\cline{1-2}
module\+\_\+main.\+cpp   &Fornecerá a interoperabilidade com o Python   \\\cline{1-2}
\end{longtabu}


Novas funcionalidades sempre deverão ser construídas dentro de novas pastas 